% Created 2025-10-03 Fri 18:00
% Intended LaTeX compiler: lualatex
\documentclass[a4paper,12pt]{article}
\usepackage{amsmath}
\usepackage{fontspec}
\usepackage{graphicx}
\usepackage{longtable}
\usepackage{wrapfig}
\usepackage{rotating}
\usepackage[normalem]{ulem}
\usepackage{capt-of}
\usepackage{hyperref}
\usepackage{luacode}
\usepackage[french]{babel}
\usepackage{microtype}
\usepackage[autolanguage]{numprint}
\npthousandsep{~}
\usepackage{fontspec}
\usepackage{ulem}
\usepackage{soul}
\setmainfont{Source Serif 4}[Path=/home/anthea/org/fonts/Source_Serif_4/static/, UprightFont=SourceSerif4-Regular.ttf, ItalicFont=SourceSerif4-Italic.ttf, BoldFont=SourceSerif4-Bold.ttf, BoldItalicFont=SourceSerif4-BoldItalic.ttf]
\setsansfont{Source Sans 3}[Path=/home/anthea/org/fonts/Source_Sans_3/static/, UprightFont=SourceSans3-Regular.ttf, ItalicFont=SourceSans3-Italic.ttf, BoldFont=SourceSans3-Bold.ttf, BoldItalicFont=SourceSans3-BoldItalic.ttf]
\setmonofont{Source Code Pro}[Path=/home/anthea/org/fonts/Source_Code_Pro/static/, UprightFont=SourceCodePro-Regular.ttf, ItalicFont=SourceCodePro-Italic.ttf, BoldFont=SourceCodePro-Bold.ttf, BoldItalicFont=SourceCodePro-BlackItalic.ttf]
\renewcommand{\familydefault}{\sfdefault}
\renewcommand{\tiny}{\small}
\renewcommand{\scriptsize}{\small}
\usepackage[usenames,dvipsnames,svgnames,table]{xcolor}
\definecolor{customgray}{HTML}{505050}
\usepackage[top=3.2cm, bottom=3.2cm, left=2.4cm, right=2.4cm]{geometry}
\usepackage{setspace,fancyhdr,indentfirst,lastpage,datetime,authblk,ifthen,etoolbox,titling}
\singlespacing
\pagestyle{fancy}
\fancyhf{}
\fancyfoot[C]{\thepage\ / \pageref{LastPage}}
\renewcommand{\headrulewidth}{0pt}
\setlength{\parindent}{0pt}
\setcounter{secnumdepth}{3}
\setlength{\columnsep}{0.8cm}
\setlength{\marginparwidth}{1.6cm}
\setcounter{page}{1}
\usepackage{csquotes}
\usepackage{array,booktabs,multirow,tabularx,colortbl,diagbox,makecell,ltablex,adjustbox,multicol}
\usepackage{enumitem}\setlist{nosep}\setlist[itemize]{leftmargin=*}
\usepackage[toc,page]{appendix}
\usepackage[nottoc]{tocbibind}
\newenvironment{keyword}{\begin{trivlist}\item[]{\bfseries Mots-clés :}}{\end{trivlist}}
\usepackage{graphicx,caption,wrapfig}
\usepackage[most,breakable,xparse,listings,skins]{tcolorbox}
\usepackage[colorinlistoftodos]{todonotes}
\usepackage{newfloat}
\DeclareFloatingEnvironment[fileext=lol,listname={\vspace{-2em}},name=Listing]{listing}
\captionsetup{format=plain,font=small,labelfont=bf}
\captionsetup[listing]{labelfont=bf,textfont=it}
\usepackage{fvextra,amsfonts,amssymb,amsmath,mathrsfs,mathtools,stmaryrd}
\usepackage{algorithm2e}
\usepackage{pgf,tikz,pgfplots,pgfplotstable,arydshln,subcaption,forest}
\pgfplotsset{compat=1.18}
\usepackage[acronym]{glossaries}
\makenoidxglossaries
\usepackage{url,orcidlink,hyperref}
\hypersetup{colorlinks=true, linkcolor=customgray, citecolor=customgray, urlcolor=customgray, pdfborder={0 0 0}, unicode=true}
\setcounter{secnumdepth}{3}
\author{PA156562}
\date{\today}
\title{}
\hypersetup{
 pdfauthor={PA156562},
 pdftitle={},
 pdfkeywords={},
 pdfsubject={},
 pdfcreator={},
 pdflang={French}}

% Setup for code blocks [1/2]

\usepackage{fvextra}

\fvset{%
  commandchars=\\\{\},
  highlightcolor=white!95!black!80!blue,
  breaklines=true,
  breaksymbol=\color{white!60!black}\tiny\ensuremath{\hookrightarrow}}

% Make line numbers smaller and grey.
\renewcommand\theFancyVerbLine{\footnotesize\color{black!40!white}\arabic{FancyVerbLine}}

\usepackage{xcolor}

% In case engrave-faces-latex-gen-preamble has not been run.
\providecolor{EfD}{HTML}{f7f7f7}
\providecolor{EFD}{HTML}{28292e}

% Define a Code environment to prettily wrap the fontified code.
\usepackage[breakable,xparse]{tcolorbox}
\DeclareTColorBox[]{Code}{o}%
{colback=EfD!98!EFD, colframe=EfD!95!EFD,
  fontupper=\footnotesize\setlength{\fboxsep}{0pt},
  colupper=EFD,
  IfNoValueTF={#1}%
  {boxsep=2pt, arc=2.5pt, outer arc=2.5pt,
    boxrule=0.5pt, left=2pt}%
  {boxsep=2.5pt, arc=0pt, outer arc=0pt,
    boxrule=0pt, leftrule=1.5pt, left=0.5pt},
  right=2pt, top=1pt, bottom=0.5pt,
  breakable}

% Support listings with captions
\usepackage{float}
\floatstyle{plain}
\newfloat{listing}{htbp}{lst}
\newcommand{\listingsname}{Listing}
\floatname{listing}{\listingsname}
\newcommand{\listoflistingsname}{List of Listings}
\providecommand{\listoflistings}{\listof{listing}{\listoflistingsname}}


% Setup for code blocks [2/2]: syntax highlighting colors

\newcommand\efstrut{\vrule height 2.1ex depth 0.8ex width 0pt}
\definecolor{EFD}{HTML}{000000}
\definecolor{EfD}{HTML}{ffffff}
\newcommand{\EFD}[1]{\textcolor{EFD}{#1}} % default
\definecolor{EFvp}{HTML}{000000}
\newcommand{\EFvp}[1]{\textcolor{EFvp}{#1}} % variable-pitch
\definecolor{EFh}{HTML}{7f7f7f}
\newcommand{\EFh}[1]{\textcolor{EFh}{#1}} % shadow
\definecolor{EFsc}{HTML}{228b22}
\newcommand{\EFsc}[1]{\textcolor{EFsc}{\textbf{#1}}} % success
\definecolor{EFw}{HTML}{ff8e00}
\newcommand{\EFw}[1]{\textcolor{EFw}{\textbf{#1}}} % warning
\definecolor{EFe}{HTML}{ff0000}
\newcommand{\EFe}[1]{\textcolor{EFe}{\textbf{#1}}} % error
\definecolor{EFl}{HTML}{ff0000}
\newcommand{\EFl}[1]{\textcolor{EFl}{#1}} % link
\definecolor{EFlv}{HTML}{ff0000}
\newcommand{\EFlv}[1]{\textcolor{EFlv}{#1}} % link-visited
\definecolor{EFhi}{HTML}{ff0000}
\newcommand{\EFhi}[1]{\textcolor{EFhi}{#1}} % highlight
\definecolor{EFc}{HTML}{b22222}
\newcommand{\EFc}[1]{\textcolor{EFc}{#1}} % font-lock-comment-face
\definecolor{EFcd}{HTML}{b22222}
\newcommand{\EFcd}[1]{\textcolor{EFcd}{#1}} % font-lock-comment-delimiter-face
\definecolor{EFs}{HTML}{8b2252}
\newcommand{\EFs}[1]{\textcolor{EFs}{#1}} % font-lock-string-face
\definecolor{EFd}{HTML}{8b2252}
\newcommand{\EFd}[1]{\textcolor{EFd}{#1}} % font-lock-doc-face
\definecolor{EFm}{HTML}{008b8b}
\newcommand{\EFm}[1]{\textcolor{EFm}{#1}} % font-lock-doc-markup-face
\definecolor{EFk}{HTML}{9370db}
\newcommand{\EFk}[1]{\textcolor{EFk}{#1}} % font-lock-keyword-face
\definecolor{EFb}{HTML}{483d8b}
\newcommand{\EFb}[1]{\textcolor{EFb}{#1}} % font-lock-builtin-face
\definecolor{EFf}{HTML}{0000ff}
\newcommand{\EFf}[1]{\textcolor{EFf}{#1}} % font-lock-function-name-face
\definecolor{EFv}{HTML}{a0522d}
\newcommand{\EFv}[1]{\textcolor{EFv}{#1}} % font-lock-variable-name-face
\definecolor{EFt}{HTML}{228b22}
\newcommand{\EFt}[1]{\textcolor{EFt}{#1}} % font-lock-type-face
\definecolor{EFo}{HTML}{008b8b}
\newcommand{\EFo}[1]{\textcolor{EFo}{#1}} % font-lock-constant-face
\definecolor{EFwr}{HTML}{ff0000}
\newcommand{\EFwr}[1]{\textcolor{EFwr}{\textbf{#1}}} % font-lock-warning-face
\newcommand{\EFnc}[1]{#1} % font-lock-negation-char-face
\definecolor{EFpp}{HTML}{483d8b}
\newcommand{\EFpp}[1]{\textcolor{EFpp}{#1}} % font-lock-preprocessor-face
\newcommand{\EFrc}[1]{\textbf{#1}} % font-lock-regexp-grouping-construct
\newcommand{\EFrb}[1]{\textbf{#1}} % font-lock-regexp-grouping-backslash
\newcommand{\EFob}[1]{#1} % org-block
\newcommand{\EFobb}[1]{#1} % org-block-begin-line
\newcommand{\EFobe}[1]{#1} % org-block-end-line
\definecolor{EFOa}{HTML}{0000ff}
\newcommand{\EFOa}[1]{\textcolor{EFOa}{#1}} % outline-1
\definecolor{EFOb}{HTML}{a0522d}
\newcommand{\EFOb}[1]{\textcolor{EFOb}{#1}} % outline-2
\definecolor{EFOc}{HTML}{a020f0}
\newcommand{\EFOc}[1]{\textcolor{EFOc}{#1}} % outline-3
\definecolor{EFOd}{HTML}{b22222}
\newcommand{\EFOd}[1]{\textcolor{EFOd}{#1}} % outline-4
\definecolor{EFOe}{HTML}{228b22}
\newcommand{\EFOe}[1]{\textcolor{EFOe}{#1}} % outline-5
\definecolor{EFOf}{HTML}{008b8b}
\newcommand{\EFOf}[1]{\textcolor{EFOf}{#1}} % outline-6
\definecolor{EFOg}{HTML}{483d8b}
\newcommand{\EFOg}[1]{\textcolor{EFOg}{#1}} % outline-7
\definecolor{EFOh}{HTML}{8b2252}
\newcommand{\EFOh}[1]{\textcolor{EFOh}{#1}} % outline-8
\definecolor{EFhn}{HTML}{008b8b}
\newcommand{\EFhn}[1]{\textcolor{EFhn}{#1}} % highlight-numbers-number
\definecolor{EFhq}{HTML}{9370db}
\newcommand{\EFhq}[1]{\textcolor{EFhq}{#1}} % highlight-quoted-quote
\definecolor{EFhs}{HTML}{008b8b}
\newcommand{\EFhs}[1]{\textcolor{EFhs}{#1}} % highlight-quoted-symbol
\definecolor{EFrda}{HTML}{707183}
\newcommand{\EFrda}[1]{\textcolor{EFrda}{#1}} % rainbow-delimiters-depth-1-face
\definecolor{EFrdb}{HTML}{7388d6}
\newcommand{\EFrdb}[1]{\textcolor{EFrdb}{#1}} % rainbow-delimiters-depth-2-face
\definecolor{EFrdc}{HTML}{909183}
\newcommand{\EFrdc}[1]{\textcolor{EFrdc}{#1}} % rainbow-delimiters-depth-3-face
\definecolor{EFrdd}{HTML}{709870}
\newcommand{\EFrdd}[1]{\textcolor{EFrdd}{#1}} % rainbow-delimiters-depth-4-face
\definecolor{EFrde}{HTML}{907373}
\newcommand{\EFrde}[1]{\textcolor{EFrde}{#1}} % rainbow-delimiters-depth-5-face
\definecolor{EFrdf}{HTML}{6276ba}
\newcommand{\EFrdf}[1]{\textcolor{EFrdf}{#1}} % rainbow-delimiters-depth-6-face
\definecolor{EFrdg}{HTML}{858580}
\newcommand{\EFrdg}[1]{\textcolor{EFrdg}{#1}} % rainbow-delimiters-depth-7-face
\definecolor{EFrdh}{HTML}{80a880}
\newcommand{\EFrdh}[1]{\textcolor{EFrdh}{#1}} % rainbow-delimiters-depth-8-face
\definecolor{EFrdi}{HTML}{887070}
\newcommand{\EFrdi}[1]{\textcolor{EFrdi}{#1}} % rainbow-delimiters-depth-9-face
\definecolor{EFany}{HTML}{CDCD00}
\newcommand{\EFany}[1]{\textcolor{EFany}{#1}} % ansi-color-yellow
\definecolor{EFanr}{HTML}{CD0000}
\newcommand{\EFanr}[1]{\textcolor{EFanr}{#1}} % ansi-color-red
\definecolor{EFanb}{HTML}{000000}
\newcommand{\EFanb}[1]{\textcolor{EFanb}{#1}} % ansi-color-black
\definecolor{EFang}{HTML}{00CD00}
\newcommand{\EFang}[1]{\textcolor{EFang}{#1}} % ansi-color-green
\definecolor{EFanB}{HTML}{0000EE}
\newcommand{\EFanB}[1]{\textcolor{EFanB}{#1}} % ansi-color-blue
\definecolor{EFanc}{HTML}{00CDCD}
\newcommand{\EFanc}[1]{\textcolor{EFanc}{#1}} % ansi-color-cyan
\definecolor{EFanw}{HTML}{E5E5E5}
\newcommand{\EFanw}[1]{\textcolor{EFanw}{#1}} % ansi-color-white
\definecolor{EFanm}{HTML}{CD00CD}
\newcommand{\EFanm}[1]{\textcolor{EFanm}{#1}} % ansi-color-magenta
\definecolor{EFANy}{HTML}{EEEE00}
\newcommand{\EFANy}[1]{\textcolor{EFANy}{#1}} % ansi-color-bright-yellow
\definecolor{EFANr}{HTML}{EE0000}
\newcommand{\EFANr}[1]{\textcolor{EFANr}{#1}} % ansi-color-bright-red
\newcommand{\EFANb}[1]{#1} % ansi-color-bright-black
\definecolor{EFANg}{HTML}{00EE00}
\newcommand{\EFANg}[1]{\textcolor{EFANg}{#1}} % ansi-color-bright-green
\definecolor{EFANB}{HTML}{0000FF}
\newcommand{\EFANB}[1]{\textcolor{EFANB}{#1}} % ansi-color-bright-blue
\definecolor{EFANc}{HTML}{00EEEE}
\newcommand{\EFANc}[1]{\textcolor{EFANc}{#1}} % ansi-color-bright-cyan
\newcommand{\EFANw}[1]{#1} % ansi-color-bright-white
\newcommand{\EFANm}[1]{#1} % ansi-color-bright-magenta
\usepackage[style=backend=biber,style=iso-numeric,citestyle=numeric-comp,doi=true,isbn=true,mincrossrefs=2,autocite=superscript]{biblatex}
\addbibresource{~/org/references.bib}
\begin{document}

\newgeometry{top=2.2cm, bottom=2.2cm, left=1.8cm, right=1.8cm}

\begin{titlepage}

\begin{minipage}[t]{0cm}
\vglue0.0cm
\includegraphics[scale=.725]{./logo/logos.png}
\end{minipage}

\begin{center}
\section*{Thèse de doctorat}
\label{sec:orgd074c51}
\vspace*{-6pt}
\section*{Pour obtenir le grade de Docteur de}
\label{sec:org8a166fe}
\vspace*{-6pt}
\section*{l'UNIVERSITE POLYTECHNIQUE HAUTS-DE-FRANCE}
\label{sec:org48c6927}
\vspace*{-6pt}
\section*{et de l'INSA HAUTS-DE-FRANCE}
\label{sec:org676d5c4}
Discipline, spécialité selon la liste des spécialités pour lesquelles l'Ecole Doctorale est accréditée :
\vspace*{-12pt}
\subsubsection*{Informatique et applications}
\label{sec:org25e54cf}
\vspace*{12pt}
\subsection*{Présentée et soutenue par Cyprien PIERRE \orcidlink{0009-0009-9040-6795}}
\label{sec:orgb2519ff}
\subsection*{Le JJ/MM/2028, à Valenciennes}
\label{sec:orgc4f0af1}
\end{center}
\subsubsection*{Ecole doctorale :}
\label{sec:org3274117}
\vspace*{-6pt}

Ecole Doctorale Polytechnique Hauts-de-France (ED PHF n°635)
\subsubsection*{Unité de recherche :}
\label{sec:orgf648a2d}
\vspace*{-6pt}

Laboratoire d'Automatique, de Mécanique et d'Informatique Industrielles et Humaines (LAMIH - UMR CNRS 8201)

\begin{center}
\section*{Systématisation de la remontée de conformité en ingénierie de la construction par approche d'interaction humain-machine sensible aux contraintes}
\label{sec:orgd6b3d06}
\vspace*{12pt}
\subsection*{JURY}
\label{sec:orgee28151}
\vspace*{-12pt}
\end{center}
\begin{multicols}{2}
\paragraph*{Président du jury}
\label{sec:org99c5749}
\begin{itemize}
\item Nom, Prénom. Titre, fonction. Lieu d'exercice
\vspace*{-12pt}
\end{itemize}
\paragraph*{Rapporteurs}
\label{sec:org6f31316}
\begin{itemize}
\item Nom, Prénom. Titre, fonction. Lieu d'exercice.
\item Nom, Prénom. Titre, fonction. Lieu d'exercice.
\vspace*{-12pt}
\end{itemize}
\paragraph*{Examinateurs}
\label{sec:orgda46d03}
\begin{itemize}
\item Nom, Prénom. Titre, fonction. Lieu d'exercice
\item Nom, Prénom. Titre, fonction. Lieu d'exercice
\item Nom, Prénom. Titre, fonction. Lieu d'exercice
\item Nom, Prénom. Titre, fonction. Lieu d'exercice
\vspace*{-12pt}
\end{itemize}
\paragraph*{Co-directeurs de thèse}
\label{sec:orgd3d554a}
\begin{itemize}
\item Christophe KOLSKI. Professeur des universités, Université Polytechnique Hauts-de-France
\item Alexis HELOIR. Professeur des universités, Université Polytechnique Hauts-de-France
\vspace*{-12pt}
\end{itemize}
\paragraph*{Membres invités}
\label{sec:orga702802}
\begin{itemize}
\item Nom, Prénom. Titre, fonction. Lieu d'exercice
\item Nom, Prénom. Titre, fonction. Lieu d'exercice
\end{itemize}

\end{multicols}
\end{titlepage}
\restoregeometry
\clearpage
\section*{Remerciements}
\label{sec:orgd75a80c}
Remercier :
\begin{enumerate}
\item Christophe Kolski
\item Alexis Heloir
\item Mathieu Chapel
\end{enumerate}

Le LAMIH et EESF

Mes collègues et amis
\clearpage
\section*{Résumé}
\label{sec:orga2e092e}
Logoden biniou degemer mat an, penn ar bed. Pa ya frouezh gaer e, kig eviti out. Traonienn amzer gallout gador beajourien, kloc’h nec’h c’hontadenn. Diskar ar koulskoude laouen c’hardeur, ostaleri da korn. Diriaou prad klouar a bugel, bro birviñ troc’hañ. Nebeutoc’h ur kenañ eñ puñs, aet gazek gorre. Planvour arvor niverenn leun merc’her, nebeutoc’h meud hi. Plad treñ pomper traezh ar, Moel plij skuizh. Stêr Ar Gall las Malo bleunioù, kontañ Pask a. Skignañ doñjer c’hardeur endervezh davarn, godell Mellag saout.

Plouared werenn lavarout Mikael ha, war kig aval. Ar gwiskamant c’haod ouzhpenn, Santeg brudet, warlene stur. Blev degas gomz enep en, c’hoarvezout vamm digant. Keit leal marteze torgenn eured, plijadur Remengol Pederneg. Gwalenn ya envel seizh Breizh, war kleuz pe. Tavarnour dro sukr plijet anzav, bugale kregiñ ahont. Garantez kelien rumm n’eus arc’hant, ya santout fazi. Holl c’henwerzh bale Pembo anal, ouzhpenn abeg an. Doñjer gantañ tavarn kreion dispign, kaol doug uhelder. Kalet da kerkoulz ganto gar, da kambrig arvar.

Toenn an beleg a mesk, yec’hed dont skrabañ. C’haod er naon istor c’havr, soñj bleunioù war. Va tenn warnañ, a goleiñ, dad forzh patatez. Keit dorn goap mouchouer Montroulez, danvez kas vamm. Evidout sukr ehan eget ennon, ahont eviti delioù. Ael divskouarn loar peurvuiañ tabut, goulenn ar kouezhañ. Gouren nijal da aval godell, lenn ur matezh. Siminal fazi leur daou trec’h, gouel graet gwer. Doñv ur Nazer da disheol, tresañ naetaat koumoul. Feunten tog c’hroc’hen Mellag Oskaleg, an ganimp, ganeomp keit.

\begin{keyword}
Logoden, biniou, degemer mat, an, penn, ar bed.
\end{keyword}
\section*{Abstract}
\label{sec:org83a7401}
Logoden biniou degemer mat an, penn ar bed. Pa ya frouezh gaer e, kig eviti out. Traonienn amzer gallout gador beajourien, kloc’h nec’h c’hontadenn. Diskar ar koulskoude laouen c’hardeur, ostaleri da korn. Diriaou prad klouar a bugel, bro birviñ troc’hañ. Nebeutoc’h ur kenañ eñ puñs, aet gazek gorre. Planvour arvor niverenn leun merc’her, nebeutoc’h meud hi. Plad treñ pomper traezh ar, Moel plij skuizh. Stêr Ar Gall las Malo bleunioù, kontañ Pask a. Skignañ doñjer c’hardeur endervezh davarn, godell Mellag saout.

Plouared werenn lavarout Mikael ha, war kig aval. Ar gwiskamant c’haod ouzhpenn, Santeg brudet, warlene stur. Blev degas gomz enep en, c’hoarvezout vamm digant. Keit leal marteze torgenn eured, plijadur Remengol Pederneg. Gwalenn ya envel seizh Breizh, war kleuz pe. Tavarnour dro sukr plijet anzav, bugale kregiñ ahont. Garantez kelien rumm n’eus arc’hant, ya santout fazi. Holl c’henwerzh bale Pembo anal, ouzhpenn abeg an. Doñjer gantañ tavarn kreion dispign, kaol doug uhelder. Kalet da kerkoulz ganto gar, da kambrig arvar.

Toenn an beleg a mesk, yec’hed dont skrabañ. C’haod er naon istor c’havr, soñj bleunioù war. Va tenn warnañ, a goleiñ, dad forzh patatez. Keit dorn goap mouchouer Montroulez, danvez kas vamm. Evidout sukr ehan eget ennon, ahont eviti delioù. Ael divskouarn loar peurvuiañ tabut, goulenn ar kouezhañ. Gouren nijal da aval godell, lenn ur matezh. Siminal fazi leur daou trec’h, gouel graet gwer. Doñv ur Nazer da disheol, tresañ naetaat koumoul. Feunten tog c’hroc’hen Mellag Oskaleg, an ganimp, ganeomp keit.

\begin{keyword}
Logoden, biniou, degemer mat, an, penn, ar bed.
\end{keyword}
\clearpage
\section*{Table des matières}
\label{sec:orgf17d122}
\renewcommand{\contentsname}{\vspace{-2em}}
\setcounter{tocdepth}{3}
\tableofcontents

\clearpage

\setcounter{section}{-1}
\section{Introduction générale}
\label{sec:org609cab7}
\subsection{Contexte et motivation}
\label{sec:orgb756334}
\subsubsection{Sclérosité systémique des lotissements traditionnels}
\label{sec:org6754ef5}
=> Chaines de valeurs traditionnelles devenu rigides et incapable de s'adapter ou d'évoluer à cause de
      la bureaucratisation des procédés ("tamponé, double tamponé\ldots{}" Au service de la France)
      la dérive des régulations (normes, réglements\ldots{})   
      la résistance au changement des collaborateurs
      l'amnésie organisationnelle et l'obsolescence des pratiques
      la déchéance du système de confiance (limites de la preuve par la formation ou par la réputation)
=> Recherche en refonte des organisation ?
=> Identification des leviers
\subsubsection{Emergence de nouveaux acteurs}
\label{sec:org2d3b1b7}
(informatique et numérique, Les ESN dans la construction ?)
=> Nouveaux outils et moyens de production
=> Besoins, impacts et opportunités

La réalisation et la maintenance des maquettes numériques, en se contentant de se superposer aux métiers historiques de la construction, court le risque d’évoluer en une forme d’organisation autonome dont l’objectif principal est la pérennisation et le développement de son organisation \autocite{lourauAnalyseInstitutionnelleQuestion1973}. Cette tendance se manifeste d’ores et déjà par la refonte des organisations de projets qui incluent des structures dédiées au  (\protect\hyperlink{gls-1}{\label{gls-1-use-1}BIM}) et composées d’une ligne de management et d’un cadre contractuel adaptés aux seules finalités de cette discipline.
Cette transformation impacte inévitablement des cultures et attitudes historiquement adoptées par les acteurs des entreprises de la construction dont les délais serrés et les objectifs parfois antagonistes favorisent le maintien d’un statu quo au sein des organisations et des pratiques \autocite{lindbladBIMImplementationOrganisational2015,paulagordogregorioContinuiteInformationnelleDans2023}. 
\subsubsection{Discorde entre ergonomie et fonctionnalités}
\label{sec:orgf9adfbc}
Objectif : définir le besoin de simplification, de convergence et d'apport de soin dans l'expérience utilisateur
Ergonomie des interfaces, ergonomie des flux et procédures, charge cognitive \& co
=> Inéficience des ergonomies applicatives et des expériences utilisateurs, dégradées au profit d'une inflation de fonctionnalités
=> Besoin de retrouver de l'abstraction

Après avoir simplifié sa structure, simplifier ses outils et monter en compétences

Ici sourcer : Blender > All car "all-in-one" un peu moins bien c'est mieux que des verticales très maitrisés mais une absence d'interopérabilité => perte de valeur dû à la non continuité des informations, la perte de contexte, etc.
Idem possible : Notion vs MS365, Revit vs AutoCAD et ses "flavours"
etc.

Explorer les bonnes pratiques en IHM, Ui, Ux, définir les "prérequis" 
Explorer le Behavior Driven Design 
\subsubsection{Conclusion}
\label{sec:orge933a51}
Faire table rase !


Expliquer le besoin de rééquilibrer les responsabilités et d'assainir la base avant de construire, notion de refondation de la chaine de valeur.

Objectif : proposer un cadre de travail scalable à forte valeur ajoutée et identification des rôles et périmètres 
Prérequis avant toute tentative de digitalisation (en 1 : on se remet en question et on balaie devant sa porte)
\begin{itemize}
\item spécialisations horizontales versus verticale
\item parcours de carrières (Expertise, Management, Projet)
\end{itemize}

Explorer la décentralisation de la confiance notamment à travers les ZKP
\subsection{Problématique de recherche}
\label{sec:orga55839d}
Question principale :

Questions complémentaires :

Comment créer un environnement de gestion des contraintes hétéroclytes ?

Comment décrire une contrainte en langage naturel ?
\subsection{Objectifs et contributions}
\label{sec:org1456414}

\subsection{Organisation du document}
\label{sec:orge2d4d4c}
\clearpage
\section{Exploration sectorielle}
\label{sec:org0ad814a}
\subsection{Introduction}
\label{sec:org6b76517}

(Okoli, Tranfield)
\begin{itemize}
\item \textbf{\textbf{Contexte et problématique}} : préciser le champ disciplinaire et la pertinence pratique/organisationnelle.
\item \textbf{\textbf{Objectif scientifique}} : situer la revue comme méthode de recherche en soi, permettant de cartographier un champ et de développer une contribution conceptuelle (typologie, cadre théorique, taxonomie, agenda de recherche).
\item \textbf{\textbf{Pertinence managériale}} : expliquer en quoi la revue éclaire les besoins des organisations et des acteurs.
\end{itemize}
\subsection{Research protocol}
\label{sec:org013456f}
\subsubsection{3.1 Research questions}
\label{sec:org733dedc}
\begin{description}
\item[{RQ1}] 

\item[{RQ2}] 
\end{description}
\subsubsection{Methode SPIDER}
\label{sec:orga5c046f}
\begin{itemize}
\item \textbf{\textbf{Sample (S)}} : acteurs, organisations, secteurs étudiés (ex. entreprises, managers, équipes projets).
\item \textbf{\textbf{Phenomenon of Interest (PI)}} : pratiques, processus, technologies, comportements managériaux étudiés.
\item \textbf{\textbf{Design (D)}} : types de designs méthodologiques inclus (études de cas, enquêtes, analyses qualitatives, etc.).
\item \textbf{\textbf{Evaluation (E)}} : indicateurs ou dimensions étudiées (performance, adoption, impacts organisationnels).
\item \textbf{\textbf{Research type (R)}} : types de recherche acceptés (empirique, théorique, revue existante).
\end{itemize}
\subsubsection{Research string}
\label{sec:orgd6904d0}
\begin{itemize}
\item Construction des équations avec opérateurs booléens et synonymes.
\item Exemple générique :
\end{itemize}

\begin{listing}[htbp]
\begin{Code}
\begin{Verbatim}
\color{EFD}("knowledge management" OR "organizational learning") AND ("digital transformation" OR "IT adoption")
\end{Verbatim}
\end{Code}
\caption{Requête SPIDER générique}
\end{listing}
\subsubsection{Targetted databases}
\label{sec:org85e4e24}
\begin{table}[htbp]
\caption{Déclinaison de la requête SPIDER par plateforme}
\centering
\begin{tabular}{ll}
Name & Query\\
\hline
Scopus & \\
Web of Science & \\
Business Source Complete (EBSCO) & \\
ScienceDirect & \\
Google Scholar (contrôle des biais) & \\
\end{tabular}
\end{table}

Littérature grise
\begin{itemize}
\item Rapports professionnels, thèses, working papers.
\item Justification de l’inclusion ou exclusion.
\end{itemize}
\subsubsection{Processus de sélection}
\label{sec:orgbb8ec7d}
(Tranfield)
Étapes
\begin{enumerate}
\item \textbf{\textbf{Recherche initiale}} → collecte des références.
\item \textbf{\textbf{Déduplication}}.
\item \textbf{\textbf{Screening par titre et résumé}}.
\item \textbf{\textbf{Screening par texte intégral}}.
\item \textbf{\textbf{Validation inter-évaluateurs}} (au moins deux chercheurs, résolution des désaccords par consensus).
\end{enumerate}

Critères d’inclusion/exclusion
\begin{itemize}
\item Inclusion : articles académiques en gestion/SHS, période temporelle définie, pertinence thématique.
\item Exclusion : non revu par les pairs (sauf gris justifié), hors champ, doublons.
\end{itemize}
\subsubsection{Formulaire d’extraction}
\label{sec:orgf97e483}
(Okoli)
\begin{itemize}
\item Identifiant (ID)
\item Référence bibliographique
\item Contexte (secteur, pays, type d’organisation)
\item Méthodologie de l’étude
\item Résultats principaux
\item Concepts/variables mobilisés
\item Contribution théorique ou pratique
\end{itemize}
\subsubsection{Évaluation de la qualité}
\label{sec:org0ecfce2}
(Okoli)
\begin{itemize}
\item Pertinence théorique (forte/moyenne/faible).
\item Validité méthodologique (forte/moyenne/faible).
\item Clarté de la contribution.
\end{itemize}
\subsubsection{Schéma de sélection}
\label{sec:org19f19e3}
Nombre d’articles identifiés, filtrés, exclus, inclus.

\textbf{PRISMA 2020 flow diagram} for new systematic reviews which included searches of databases and registers only
\begin{itemize}
\item Présenter le flux : articles identifiés, retenus, exclus, inclus.
\item Fournir la checklist 2020 PRISMA-RR (traçabilité).
\item Appliquer les interim guidance pour les Rapid Reviews issues du groupe Cochrane : expliciter les écarts méthodologiques, les raccourcis, la justification de ces choix.
\item Inclure un diagramme de flux (identification → sélection → inclusions) adapté au contexte RR.
\item Intégrer les éléments de publication / éthique : auteurs, contributions, relecteurs, conflits d’intérêt.
\item Mention explicite du fait que PRISMA-RR est en développement et que ce rapport est conforme aux principes provisoires.
\end{itemize}
\url{https://pmc.ncbi.nlm.nih.gov/articles/PMC12013547}
\url{https://pubmed.ncbi.nlm.nih.gov/39038926}
\url{https://www.equator-network.org/wp-content/uploads/2018/02/PRISMA-RR-protocol.pdf}
\subsection{Analysis and results}
\label{sec:orgfda1f08}
\begin{itemize}
\item \textbf{\textbf{Analyse descriptive}} : nombre d’articles, évolution temporelle, répartition par journaux/méthodes.
\item \textbf{\textbf{Analyse thématique}} : regroupement des contributions en catégories conceptuelles.
\item \textbf{\textbf{Construction conceptuelle}} : cadre, typologie, ou modèle explicatif.
\item \textbf{\textbf{Agenda de recherche}} : identification des lacunes et pistes futures.
\end{itemize}
\subsection{Discussion}
\label{sec:org189666f}
\textbf{\textbf{Synthèse des apports}} : résumé des résultats majeurs.
\textbf{\textbf{Implications théoriques}} : enrichissement du corpus scientifique en gestion.
\textbf{\textbf{Implications pratiques}} : recommandations pour les acteurs managériaux.
\textbf{\textbf{Limites méthodologiques}} : biais de sélection, couverture des bases, etc.
\textbf{\textbf{Perspectives}} : agenda pour futures recherches.
\subsection{Treats to validity}
\label{sec:org31c9529}
Risques de biais de publication.
Risques liés à l’échantillonnage ou aux bases de données.
Stratégies d’atténuation (diversification, double codage).
\subsection{Research opportunity}
\label{sec:orge6c686f}

\subsection{Conclusion}
\label{sec:org90e3c11}
\clearpage
\section{Etat de l'art}
\label{sec:org6d7ded3}
\subsection{Introduction}
\label{sec:org17f0559}

\subsubsection{Background}
\label{sec:orge68d995}
\subsubsection{Business rules}
\label{sec:org9d019ce}

\subsubsection{Ecological interface design}
\label{sec:org61080a2}
\subsection{Research protocol}
\label{sec:org37d442a}
\subsubsection{Research questions}
\label{sec:orgff5f280}
\begin{description}
\item[{RQ1}] 

\item[{RQ2}] 

\item[{RQ3}] 
\end{description}
\subsubsection{Methode PICOC}
\label{sec:org02bcb9c}
\begin{itemize}
\item Population
\item Intervention
\item Comparison
\item Outcome
\item Context
\end{itemize}
\subsubsection{Inclusion criteria}
\label{sec:orgb2d15c8}
Articles publiés entre 2010 et 2025.
Langues
Types de publications
Domaine pertinent
\subsubsection{Exclusion criteria}
\label{sec:orgc208415}
Études non revues par les pairs (sauf littérature grise explicitement incluse).
Articles incomplets, sans résultats empiriques, etc.
Doublons.
\subsubsection{Quality assessment criteria}
\label{sec:orgb11b5f8}

\subsubsection{Research strings}
\label{sec:orge0b94cd}
Construction des requêtes (mots-clés, opérateurs booléens, synonymes).

Exemple générique :
\begin{Code}
\begin{Verbatim}
\color{EFD}("machine learning" OR "deep learning") AND ("software engineering" OR "systems")
\end{Verbatim}
\end{Code}

Justification des choix de mots-clés.
\subsubsection{Taretted databases}
\label{sec:org2db8ff6}
ACM Digital Library
IEEE Xplore
Scopus
Web of Science
Autres : …
\subsubsection{Processus de recherche}
\label{sec:orgfce9148}
Recherche initiale → collecte des résultats → exportation (BibTeX, CSV).
Déduplication (Zotero, EndNote, Mendeley, etc.).
\subsubsection{Processus de sélection}
\label{sec:orge3849ba}
Étape 1 : filtrage par titre et résumé.
Étape 2 : filtrage par texte intégral.
Étape 3 : validation inter-évaluateurs (au moins deux chercheurs).
\subsubsection{Formulaire d’extraction}
\label{sec:org0b72372}
Champs obligatoires :
    Identifiant (ID)
    Référence bibliographique complète
    Année de publication
    Contexte (population, domaine, technologie)
    Méthodologie de l’étude
    Résultats principaux (Outcome)
    Limites rapportées
\subsubsection{Évaluation de la qualité}
\label{sec:org985fe5b}
Checklist PRISMA, indiquer la localisation de chaque item dans le rapport final.

Checklist qualité (exemple) :
    Clarté des objectifs : oui/non
    Méthodologie décrite : oui/non
    Données empiriques disponibles : oui/non
    Validité des résultats : élevé/moyen/faible
\subsubsection{Schéma de sélection}
\label{sec:org1b14340}
Nombre d’articles identifiés, filtrés, exclus, inclus.

\textbf{PRISMA 2020 flow diagram} for new systematic reviews which included searches of databases and registers only
\subsection{Analysis and results}
\label{sec:orgca76882}
Méthodes d’analyse
    Quantitative (comptages, distributions, tendances temporelles).
    Qualitative (analyse thématique, catégorisation, taxonomie).
    Meta-analysis (si applicable).
\subsubsection{RQ1:}
\label{sec:org254a807}
\subsubsection{RQ2:}
\label{sec:orgfe09c39}
\subsubsection{RQ3:}
\label{sec:org1a4e7d3}
\subsection{Wrapping up}
\label{sec:org479158c}
\subsubsection{General discussion}
\label{sec:org6d7f504}
Contribution scientifique :
    Lacunes identifiées
    Etat de l’art consolidé.
Contribution pratique :
    Recommandations
    Implications pour les chercheurs et praticiens.
Limites méthodologiques du protocole.
\subsubsection{Recommendations}
\label{sec:org6467141}
\subsection{Treats to validity}
\label{sec:org121a95f}
Risques de biais de publication.
Risques liés à l’échantillonnage ou aux bases de données.
Stratégies d’atténuation (diversification, double codage).
\subsection{Research opportunity}
\label{sec:org1991ad4}

\subsection{Conclusion}
\label{sec:org9068110}
\clearpage
\section{Problématique de recherche}
\label{sec:org744ea89}
\subsection{Introduction}
\label{sec:org4f611fc}
Rappel : après l’état de l’art, une zone non résolue est identifiée (ex. gestion des contraintes via IHM écologiques).
Objectif : transformer cette zone non résolue en une problématique scientifique explicite.
Cadres mobilisés : Whetten, Alvesson \& Sandberg, QQOQCCP, RCA.
\subsection{Définition de la problématique}
\label{sec:orgfa1be84}
(Whetten, 1989)
What : quels éléments précis posent problème (ex. multiplicité et incohérence des contraintes projet).
How : comment ces éléments interagissent ou produisent des effets négatifs.
Why : pourquoi il est crucial d’y répondre (enjeux théoriques + pratiques).
Who / Where / When : quels acteurs, contextes, phases du projet sont concernés.
\subsection{Analyse critique}
\label{sec:org28f4f43}
(Problematization – Alvesson \& Sandberg, 2011)
Identifier les hypothèses dominantes dans la littérature (ex. “les contraintes sont gérables par les méthodes classiques de planification”).
Montrer leurs limites ou leur obsolescence.
Créer une tension : pourquoi ces hypothèses ne suffisent plus dans les environnements actuels.
Reformuler la problématique comme une contradiction non résolue.
\subsection{Déclinaison opérationnelle}
\label{sec:orge3b1225}
(QQOQCCP / 5W1H)
Quoi : description détaillée du problème organisationnel.
Qui : acteurs directement et indirectement affectés.
Où : environnement spécifique (projets complexes, systèmes socio-techniques).
Quand : temporalité critique (conception, exécution, suivi).
Comment : limites des solutions actuelles.
Combien : ampleur mesurée (coûts, délais, incidents).
Pourquoi : justification du caractère central du problème.
\subsection{Analyse des causes profondes}
\label{sec:org2f190a5}
(RCA / Ishikawa / 5 Why’s)
Identification des sources techniques, organisationnelles, cognitives.
Mise en évidence de l’origine structurelle du problème : absence de cadre unifié pour gérer, résoudre et préserver les contraintes.
\subsection{Proposition de recherche}
\label{sec:org9fb5feb}
Formulation claire, nette et précise de la problématique en une phrase :
« Comment concevoir une pratique organisationnelle et un cadre outillé permettant de modéliser, résoudre et préserver les contraintes dans les projets complexes, en intégrant des interfaces écologiques adaptées aux acteurs ? »

Positionner cette formulation comme le pivot entre état de l’art et théorisation.
\subsubsection{Questions de recherche}
\label{sec:orgdaa4da3}
Question principale : Comment développer une approche d'ingénierie par les contraintes pour améliorer la conception et la validation des systèmes de génie électrique ?

Questions secondaires :
\begin{itemize}
\item Quels mécanismes de vérification formelle intégrer dans cette approche ?
\item Comment remonter aux utilisateurs [\ldots{}] (IHM)
\item Comment assurer la traçabilité des contraintes techniques ?
\item Quelle est l'efficacité de cette approche comparée aux méthodes traditionnelles ?
\end{itemize}
\subsubsection{Approche méthodologique}
\label{sec:org8dbb1e0}
\subsection{Discussion}
\label{sec:org916939f}
Montrer que la problématique n’est pas une simple lacune, mais une tension théorique + enjeu pratique majeur.
Mettre en évidence la valeur de cette problématique pour :
les chercheurs (nouvelle théorie),
les praticiens (nouveaux outils).
\subsection{Conclusion}
\label{sec:orgc01c3a9}
Récapitulatif de la problématique formalisée.
Insistance sur son rôle structurant pour la suite (théorisation → implémentation → évaluation).
\clearpage
\section{Fondements théoriques}
\label{sec:orgbaecd31}
\subsection{Méthodologie}
\label{sec:org5ff4f98}
\subsubsection{Stratégie de recherche}
\label{sec:orgadede49}
La stratégie repose sur la Design Science Research (DSR) (Hevner et al., 2004 ; Peffers et al., 2007 ; Gregor \& Jones, 2007) comme ancrage principal pour construire une théorie de conception en gestion de projet.
Elle est enrichie par :
\begin{itemize}
\item l’Action Design Research (ADR) (Sein et al., 2011) afin d’intégrer les utilisateurs dans la boucle de recherche,
\item le cadre CIMO / Realist Evaluation (Pawson \& Tilley, 1997 ; Denyer et al., 2008) pour structurer les mécanismes explicatifs,
\item l’approche processuelle de Langley (1999) pour représenter la dynamique organisationnelle et ses évolutions.
\end{itemize}
\subsubsection{Socles théoriques}
\label{sec:org636cca7}
Théories des artefacts de conception et de l’action (DSR, ADR).
Théories causales mécanistes (CIMO).
Théorisation processuelle (Langley).
Bases en sciences de gestion : routines organisationnelles (Pentland \& Feldman), Knowledge Management (Nonaka, Davenport), écologie des interfaces (Vicente), théorie des jeux, graphes.
\subsubsection{Organisation}
\label{sec:orgf5087fa}
La méthodologie est organisée en quatre parties :
\begin{itemize}
\item Définition des objectifs (DSR + ADR)
\item Principes de conception (DSR)
\item Mécanismes causaux (CIMO)
\item Dynamiques processuelles (Langley)
\end{itemize}
\subsection{Définition des objectifs}
\label{sec:org65ef4e4}
\begin{itemize}
\item Structurer les objectifs de recherche en six étapes (problem identification, objectifs, design, démonstration, évaluation, communication) \autocite[et al. (2007)]{Peffers}.
\item Garantir la pertinence (problèmes issus du terrain) et la rigueur scientifique (bases théoriques) \autocite[et al. (2004)]{Hevner}.
\item Formuler les objectifs conjointement avec les praticiens, en laboratoire puis en contexte industriel \autocite[et al. (2011) (ADR)]{Sein}.
\end{itemize}
\subsection{Principes de conception}
\label{sec:org718c12e}
Structuration des principes \autocite[\& Jones (2007)]{Gregor} :
\begin{itemize}
\item But \& portée : Développer une pratique organisationnelle de gestion de projet qui permet de modéliser, résoudre et préserver les contraintes, tout en assurant traçabilité, faisabilité et adaptation continue.
\item Constructs : Acteur, rôle, objet-projet, contrainte, test, graphe de dépendances, mécanisme de propagation, événement, log, interface écologique.
\item Principes de forme et de fonction (unicité, traçabilité, feedback en temps réel)
\begin{itemize}
\item Unicité : une contrainte exprimée en langage contrôlé correspond à une méthode exécutable unique.
\item Traçabilité totale : lien continu du texte utilisateur jusqu’aux logs d’exécution.
\item Écologie de l’interface : feedback visuel et immédiat des contraintes et écarts.
\end{itemize}
\item Principes d’implémentation (pipeline CNL → Graphe → Solveur → IHM)
\item Traduction des objectifs en principes testables et en artefacts concrets (modèles, prototypes).
\item Intégration de la co-construction (ADR) pour que ces principes soient ajustés en continu avec les utilisateurs.
\end{itemize}

Justificatory knowledge : S’appuie sur la littérature en routines organisationnelles, gestion des connaissances (SECI), écologie des interfaces (EID), planification par contraintes (TOC/PERT/CPM), théorie des jeux et graphes.
\subsection{Mécanismes causaux}
\label{sec:org117d37a}
Utilisation du cadre CIMO (Denyer et al., 2008 ; Pawson \& Tilley, 1997) pour exprimer les mécanismes :
\begin{itemize}
\item Contexte (type de projet, maturité organisationnelle)
\item Intervention (instanciation du graphe de contraintes, IHM écologique)
\item Mécanisme (propagation, négociation, préservation via logs/tests)
\item Outcome (réduction du temps de résolution, meilleure conformité, traçabilité accrue)
\end{itemize}

Exemple
\begin{verbatim}
Dans un projet à forte complexité contractuelle (C), l’instanciation d’un pipeline CNL→Graphe (I) active le mécanisme de détection précoce des conflits (M), réduisant le temps moyen de résolution (O).
\end{verbatim}
\subsection{Dynamiques processuelles}
\label{sec:orgd8311ec}
Théoriser les dynamiques organisationnelles \autocite[(1999)]{Langley}.
Méthodes utilisées :
\begin{itemize}
\item Temporal bracketing (séquençage des phases contraintes/tests).
\item Visual mapping (diagrammes  (\protect\hyperlink{gls-2}{\label{gls-2-use-1}UML}),  (\protect\hyperlink{gls-3}{\label{gls-3-use-1}SysML}),  (\protect\hyperlink{gls-4}{\label{gls-4-use-1}BPMN}) pour modéliser les processus).
\item Narrative strategies (construction d’histoires organisationnelles reliant données empiriques et mécanismes théorisés).
\end{itemize}

Apport : démonstration que la pratique organisationnelle évolue par itérations, au-delà d’une simple modélisation statique.
\subsection{Formulation finalisée de la théorie}
\label{sec:org14abe8b}
(Design Theory Statement)

Conformément à Gregor \& Jones (2007), la formulation finale de la théorie issue de l’étude peut être structurée en huit éléments. Ci-dessous un exemple de rédaction, que vous raffinerez ensuite à partir des résultats empiriques :

Proposition simple :
\begin{verbatim}
When working on a complex project, any actor will benefit from an ecological HCI design to manage their business rules.
\end{verbatim}

Théorie :
\begin{verbatim}
In complex projects (Where), actors with decision or coordination roles (Who), when provided with an ecological HCI (How) integrating executable business rules (What), during planning and monitoring phases (When), will experience a measurable reduction in resolution time (How much) and increased compliance (Outcome), because mechanisms of visibility, propagation, and traceability (Why) support better coordination across actors (CIMO).
\end{verbatim}
\subsection{Discussion}
\label{sec:org5755b10}

\subsection{Conclusion}
\label{sec:org380f743}
Le chapitre aboutit à une architecture de théorisation hybride et multi-niveaux :
Macro-niveau (DSR/ADR) : définition et validation itérative d’une design theory.
Mésos-niveau (CIMO) : formalisation des mécanismes causaux.
Micro-niveau (Langley) : représentation des processus et routines dans le temps.
\clearpage
\section{Implémentation}
\label{sec:orgacf1b19}
\subsection{Architecture logicielle}
\label{sec:orgecfd6f3}
\subsubsection{Choix technologiques}
\label{sec:orgae546d3}
\subsubsection{Modules principaux}
\label{sec:org61fa989}
\subsubsection{Tests unitaires et d'intégration}
\label{sec:org94580ef}
\subsection{Conclusion}
\label{sec:org3907dff}
\clearpage
\section{Validation expérimentale}
\label{sec:orga72a628}
\subsection{Protocole d'essais}
\label{sec:orge98a348}

\subsection{Cas d'étude 1 :}
\label{sec:orgda6ad2c}
\subsection{Cas d'étude 2 :}
\label{sec:org824cf16}
\subsection{Cas d'étude 3 :}
\label{sec:orgdb22885}

\subsection{Analyse comparative}
\label{sec:org1e3ba92}
\subsubsection{Métriques de performance}
\label{sec:orge1adc19}
Comment mesurerer l'efficacité ? Temps gagné ? Nombre d'erreurs détectées en amont ?
Envisager des questionnaires pour évaluer la charge mentale (ex: NASA-TLX) avant et après l'utilisation de votre outil.
\subsubsection{Limitations identifiées}
\label{sec:org44ec4fa}
\subsection{Conclusion}
\label{sec:org4300c60}
\clearpage
\section{Discussion et perspectives}
\label{sec:orga2625e2}
\subsection{Analyse des contributions}
\label{sec:org319f25e}
\subsubsection{Contributions théoriques}
\label{sec:orgadd8bfe}
\subsubsection{Contributions méthodologiques}
\label{sec:orgaa004c8}
\subsubsection{Contributions pratiques}
\label{sec:orge19e81b}
\subsection{Limites et défis}
\label{sec:org7fecc73}
\subsubsection{Limites théoriques}
\label{sec:org65b7858}
\subsubsection{Limites pratiques}
\label{sec:org8e4b8da}
\subsubsection{Défis organisationnels}
\label{sec:orgbc92458}
\subsection{Perspectives d'amélioration}
\label{sec:org7bebbd0}
\subsubsection{Extensions théoriques}
\label{sec:org7a9bd6a}

\begin{figure}[htbp]
\centering
\includegraphics[width=.9\linewidth]{./svg/long-term-goal.pdf}
\caption{\label{fig:org20b66c3}Vers une ingénierie sans confiance ?}
\end{figure}
\subsubsection{Améliorations techniques}
\label{sec:org1e31b24}
\subsubsection{Extensions domaines}
\label{sec:org62a05f4}
\subsection{Impact scientifique et industriel}
\label{sec:org7be5b2a}
\subsubsection{Impact sur la recherche}
\label{sec:org720490e}
\subsubsection{Impact industriel}
\label{sec:org16c6c14}
\subsubsection{Impact sociétal}
\label{sec:org3f06f61}
Discuter de la faisabilité et des implications de la refonte de la filière.
\clearpage
\section{Conclusion générale}
\label{sec:org83e2493}
Synthèse des contributions

Contribution théorique majeure

Innovation méthodologique

Validation expérimentale

Réponse à la question principale

Réponse à la questions secondaires

Perspectives d'avenir


\begin{quote}
{[}!Info] Commentaire prospectif sur l'ouvrage et ses conclusions
\end{quote}

Espérer une évolution des plateformes d'accès aux normes (cobaz) pour simplifier la configuration des environnements de travail (NF EN etc. et gestion des exigences)

L'avenir, un terrain fertile pour l'ingénierie intégrée ?

Les futures ruptures technologiques (?)
\clearpage
\section{Références du document}
\label{sec:org01ed123}
\subsection{Liste des figures}
\label{sec:org5f4b181}
\renewcommand{\listfigurename}{\vspace{-2em}}
\listoffigures
\subsection{Liste des tableaux}
\label{sec:org9844751}
\renewcommand{\listtablename}{\vspace{-2em}}
\listoftables
\subsection{Liste des codes sources}
\label{sec:org707862c}
\renewcommand{\lstlistingname}{\vspace{-2em}}
\lstlistoflistings
\subsection{Liste des glosses}
\label{sec:orgabc2c94}


\subsection{Liste des acronymes}
\label{sec:org3b13804}
\textbf{\hypertarget{gls-79}{BPMN}}\hspace*{1em}\hspace*{.5em}\pageref{gls-4-use-1}

\textbf{\hypertarget{gls-70}{BIM}}\hspace*{1em}\hspace*{.5em}\pageref{gls-1-use-1}

\textbf{\hypertarget{gls-184}{ICS}}\hspace*{1em}\hspace*{.5em}\pageref{gls-5-use-1}, \pageref{gls-5-use-2}, \pageref{gls-5-use-3}

\textbf{\hypertarget{gls-329}{SysML}}\hspace*{1em}\hspace*{.5em}\pageref{gls-3-use-1}

\textbf{\hypertarget{gls-336}{UML}}\hspace*{1em}\hspace*{.5em}\pageref{gls-2-use-1}

\clearpage
\section{Bibliographie}
\label{sec:org3184189}
\small
\printbibliography[heading=none]

\normalsize
\clearpage

\appendix
\section{Analyse des normes}
\label{sec:orgd32a055}
\subsection{Introduction}
\label{sec:org56d56b0}
\subsection{Périmètre de l'étude}
\label{sec:orgda76581}
L’étude se concentre sur les normes volontaires françaises (NF) référencées par AFNOR et publiées à la date de la collecte.
Le périmètre inclut toutes les normes relevant du domaine “Construction et urbanisme” selon la classification AFNOR Norm’Info.

Les normes ISO/IEC sont souvent transposées en normes NF (NF EN ISO, etc.)

AFNOR est le point d’entrée national reconnu par l’État pour la normalisation volontaire.

Les normes d’application obligatoire sont issues de ce corpus (via réglementations).

Par l'analyse des textes et de leurs métadonnées, nous tenterons de répondre aux questions suivantes :

\begin{itemize}
\item Q1 : Quelle est l’ampleur documentaire du corpus normatif applicable à l’industrie de la construction, mesurée en nombre de documents et en volume paginé ?
\item Q2 : La filière construction présente-t-elle une densité normative supérieure à celle d’autres secteurs industriels comparables, en termes de nombre de normes actives et de leur volumétrie documentaire ?
\item Q3 : Comment les textes normatifs se répartissent-ils entre les sous-domaines techniques, professions et spécialités représentatives de la filière construction, selon les descripteurs et indices de classement ?
\end{itemize}
\subsection{Méthodologie}
\label{sec:orgb285068}
\subsubsection{Cadre juridique}
\label{sec:orga1fd340}
Il n'existe pas de base de données publiques recenssant l'ensemble des textes de normes et leurs métadonnées. Il convient donc de constituer cette base de donnée en collectant les informations publiquement accessibles.

Cette opération implique l'emplois du webscraping.

\begin{quote}
Le webscraping consiste à extraire automatiquement (to scrape : gratter), de manière massive des données d'un site web. -- INRAE\autocite{quesnevilleRecommandationsUsagesWebscraping2024}
\end{quote}

La légalité d'une telle opération semble parfois faire débat. (ref à ajouter)

Certains acteurs, notamment l'AFNOR, s'oppose à la fouille automatisée des texte que l'organisme fournis ainsi qu'à l'emploi de modèles d'intelligence artificielle sur ceux-ci tels que l'exprime ce paragraphe apposés en première page de couverture :
\begin{quote}
AFNOR, en tant que titulaire des droits d’auteur ou distributeur autorisé, s’oppose expressément à toute intégration, transmission ou absorption totale ou partielle du présent document par des moteurs ou algorithmes d’Intelligence Artificielle (IA). AFNOR s’oppose également à toute fouille de textes et de données ou création dérivée produite par une IA et basée sur le présent document.
\end{quote}

Cela dit, le Code de la propriété intellectuelle précise les modalités de copie et de reproduction des bases de données (Art. L342-3) et les droits de manipulation des textes dans un cadre de recherche scientifique (Art. L122-5 et L122-5-3)\autocite{CodeProprieteIntellectuelle} en précisant spécifiquement que :
\begin{quote}
Des copies ou reproductions numériques d'œuvres auxquelles il a été accédé de manière \textbf{licite} peuvent être réalisées \textbf{sans autorisation des auteurs} en vue de \textbf{fouilles de textes et de données} menées à bien aux seules fins de la recherche scientifique par les organismes de recherche [\ldots{}] ou pour leur compte et à leur demande par d'autres personnes, y compris dans le cadre d'un partenariat sans but lucratif avec des acteurs privés. -- Article L122-5-3\autocite{CodeProprieteIntellectuelle}
\end{quote}

L'exploration des textes publiés par l'AFNOR et obtenus de manière licite est donc autorisée.
\subsubsection{Données collectées}
\label{sec:orgc401e10}
Métadonnées accessibles publiquement via Norm’Info et Boutique AFNOR :
    Référence (ex : NF C15-100)
    Titre
    Date de publication
    Nombre de pages
    Codes  (\protect\hyperlink{gls-5}{\label{gls-5-use-1}ICS})
    Indice de classement
    Domaine technique
    Commission de normalisation
\subsubsection{Protocole technique}
\label{sec:orgcdaefe4}
Méthode de collecte : Web scraping (à documenter)

Limite : seules les normes publiées et publiquement référencées sur le site marchand de l'AFNOR disponibles à la vente ou référencées, sont incluses.
\subsection{Méthodes d'analyse}
\label{sec:orgb01a28a}
Analyse descriptive :
    Nombre total de documents / pages
    Évolution temporelle des publications (si date disponible)

Analyse comparative :
    Densité normative dans la construction vs autres domaines AFNOR (en comparant les volumes \protect\hyperlink{gls-5}{\label{gls-5-use-2}ICS} sectoriels)

Analyse thématique / taxonomique :
    Catégorisation des normes par code \protect\hyperlink{gls-5}{\label{gls-5-use-3}ICS}, indice de classement, domaine technique
    Projection possible par métier : architecture, génie civil, thermique, électricité…

Outils recommandés : Python (pandas + matplotlib)
\subsection{Résultats obtenus}
\label{sec:org2fde983}
Cartographie de la norme dans la construction

Poids normatif par spécialité

Identification d’une sur-normativité éventuelle

Premiers indicateurs pour évaluer la « charge de la norme »
\subsection{Discussion et perspectives}
\label{sec:orgc4eb869}
\clearpage
\section{Analyse des ontologies}
\label{sec:orgfe290ed}
Echanger avec A.Vial du CSTB pour une collaboration
Application des ontologies en gestion des contraintes
Panorama et comportement
\section{Analyse des méthodes de test logiciels}
\label{sec:org2abcaa3}
\section{Analyse des langages de balisage légers}
\label{sec:orga91e022}
\section{Analyse des standards d'information}
\label{sec:org2604a8e}
\clearpage
\end{document}
